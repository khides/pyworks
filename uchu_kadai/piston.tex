\documentclass[a4paper, titlepage]{jsarticle}
\usepackage[dvipdfmx]{graphicx} % Required for inserting images
\usepackage{comment}
\usepackage{amsmath}
\usepackage[margin=20truemm]{geometry}
\usepackage{amssymb}
\usepackage{float}
\usepackage{here}
\usepackage{multicol}
\usepackage{url}

\begin{document}
\begin{table}[h]
    \caption{ピストン及びピストンピンの寸法}
    \label{table:SpeedOfLight}
    \centering
    \begin{tabular}{lllll}
       \hline
       項目       & 記号        & 標準寸法      & 寸法範囲[mm] & 決定値[mm] \\
       \hline \hline
       シリンダ内径 & D&  &  & 80\\
       ピストン全高 & L&(0.9\sim1.0)D &72\sim80 &72\\
       ピストン圧縮高さ &L_1 &(0.4\sim0.5)D &32\sim40 &32\\
       ピン下方部高さ &L_2& L − L_1 &&40\\
       トップランド高さ& L_r &(0.06\sim0.10)D &4.8\sim8.0 &4.8\\
       コンプレッションリングみぞ幅& h &(0.03\sim0.04)D &2.4\sim3.2 &2.5\\
       オイルリングみぞ幅 &h'&(0.05\sim0.10)D &4.0\sim8.0 &4.0\\
       ランド幅 &l_e &(1.0\sim1.3)h &2.5\sim3.25 &2.5\\
       ピン穴径 &d_p &(0.24\sim0.3)D &19.2\sim24 &20\\
       ピンボス外径 &d'_p& (1.4\sim1.5)d_p &28\sim30& 28\\
       \end{tabular}
   \end{table}
   \begin{table}[h]
    % \caption{ピストン及びピストンピンの寸法}
    % \label{table:SpeedOfLight}
    \centering
    \begin{tabular}{lllll}
       ピンボス間距離 &l_p &(0.3\sim0.4)D &24\sim32 &24\\
       スカート下端肉厚 &t′&(0.02\sim0.03)D &1.6\sim2.4 &1.6\\
       頭部肉厚 &t &(0.06\sim0.065)D& 4.8\sim5.2 &4.8\\
       すきま\\
       最上部 &b1 &0.8D × 10^−2 &0.64& 0.64\\
       ランド部& b'_1& 0.6D × 10^−2 &0.48 &0.48\\
       しゅう動部分上部 &b'_2&(0.4\sim0.6)D × 10^-2 &0.32\sim0.48 &0.48\\
       しゅう動部分下部 &b_2 &(0.2\sim0.3)D × 10^−2 &0.16\sim0.24 &0.24\\
       ピストン外径\\
       最上部 &D1 &D − b_1 &79.36 &79.36\\
       \end{tabular}
   \end{table}
   \begin{table}[h]
    % \caption{ピストン及びピストンピンの寸法}
    % \label{table:SpeedOfLight}
    \centering
    \begin{tabular}{lllll}
       ランド部 &D'_1 &D − b'_1 &79.52 &79.52\\
       しゅう動部分上部 &D'_2& D − b'_2 &79.52\sim79.68& 79.52\\
       しゅう動部分下部 &D_2& D − b_2 &79.76\sim79.84 &79.76\\
       ピストン上部内径 &d_r &(0.7\sim0.8)D &56\sim64 &64\\
       コンプレッションリング厚さ &T_c &&&3.35\\
       オイルリング厚さ& T_o &&&3.35\\
       コンプレッションリング内径 &A_c&D − (2T_c + 0.007D + 0.635) &&72.11\\
       オイルリング内径 &A_o &D − (2T_o + 0.006D + 1.524)&& 71.30\\
       \hline
    \end{tabular}
\end{table}

\end{document}